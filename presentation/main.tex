\documentclass[draft]{beamer}

\usetheme{Pittsburgh}
\usecolortheme{dove}

\usepackage[]{amsmath}
\usepackage{amssymb}
\usepackage{cmbright}
\usepackage{bm}

\newcommand{\RR}{\mathbb{R}}
\newcommand{\abs}[1]{|#1|}
\newcommand{\set}[1]{\{ #1 \}}
\newcommand{\defkey}{\textbf}

\title{Measuring polygonal ``niceness''}
\author[Duppala, Kraemer]{%
Sharmila Duppala\inst{1}, %
David Kraemer\inst{1}}

\institute[Stony Brook University]
{
  \inst{1}%
  AMS 545: Computational Geometry \\
  Stony Brook University
}

\date[2018]{May 1, 2018}


\begin{document}

\frame{\titlepage}

\begin{frame}[t]{Preliminaries}
  \begin{itemize}
    \item $\lambda(A)$ is the area of a set $A \subseteq \RR^2$.
    \item Let $P \subseteq \RR^2$ denote a simple bounded closed polygon.
    \item $\partial P$ is the boundary of $P$.
    \item $[x,y]$ is the closed line segment bounded by $x,y \in \RR^2$.
  \end{itemize}
\end{frame}

\begin{frame}[t]{Measures of ``niceness''}
  \begin{definition}[$\alpha$-fatness]
    The \defkey{$\bm{\alpha}$-fatness score} is given by
    \begin{equation*}
      \alpha(P) = \inf\set{
        \frac{\lambda(B(x, \rho) \cap P)}{\lambda(B(x, \rho))} :
        x \in \partial P, \rho > 0, P \not\subseteq B(x, \rho)
      }
    \end{equation*}
  \end{definition}
  \begin{itemize}
    \item For every point $x \in \partial P$, sweep through all balls of radius
      $\rho$, centered at $x$, that don't contain $P$.
    \item Compute the proportion of area covered by both $P$ and $B(x,\rho)$.
    \item Find the smallest such proportion. This is $\alpha(P)$.
    \item It's much easier when the ball is a square!
  \end{itemize}
\end{frame}

\begin{frame}[t]{Measures of ``niceness''}
  \begin{definition}[Chord-area]
    The \defkey{chord-area score} is given by
    \begin{equation*}
      s_\lambda(P) =
      \inf_{x,y \in \partial P} \max( \lambda(P'), \lambda(P'')),
    \end{equation*}
    where the chord $[x,y]$ partitions $P = P' \cup P''$.
  \end{definition}
  \begin{itemize}
    \item This is a ``minimax''-esque definition. We want the least bad resulting split.
  \end{itemize}
\end{frame}

\begin{frame}[t]{Implementation}
  \begin{itemize}
    \item The measurements were implemented in C++ using CGAL with exact
      arithmetic kernel.
    \item We used a $\delta$-boundary discretizing scheme: the length of $[x_k,
      x_{k+1}]$ is at most $\delta > 0$ for consecutive boundary vertices.
    \item Generating useful test polygons was tricky.
    \item We ran our measurements on [THESE CLASSES OF POLYGONS].
  \end{itemize}
\end{frame}

\begin{frame}[t]{Results}
  
\end{frame}

\end{document}
