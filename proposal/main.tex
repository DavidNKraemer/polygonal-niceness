\documentclass{article}
\usepackage[utf8]{inputenc}

\title{Fatness of Simple Polygons}
\author{Sharmila Duppala \and David Kraemer}
\date{April 2018}

\begin{document}

\maketitle

\section{Problem Statement} 

Computing the Niceness of a Polygonal Shape. There are various notions of
quantifying how “nice” or how “fat” a simple polygon $P$ is. A “nicest” polygon
might be a regular n-gon, which most closely approximates a circular disk. This
project seeks to implement some precise metrics for niceness, and compare them
on simple polygons (possibly moused in by a user or read in from a file, etc).
To make it simple and discrete, I propose that you discretize the boundary of
$P$
with discrete points pi (additional vertices), with a prescribed spacing,
$\delta$. (That is, for edges of $P$ longer than $\delta$, add vertices along the
edge so that the new sub-segments are of length at most $\delta$, for a
user-specified parameter $\delta$.)

Then, consider discrete choices of radii $r = \rho, 2\rho, 3\rho, \ldots$ for
disks, $B(p_i,r)$, centered at the discrete boundary points $p_i$, for each
radius $r$. The (discrete) fatness of $P$ is given by the smallest value of the
ratio $area(B(p_i,r)\cap P)/area(B(p_i, r)$, over all choices of $p_i$ and $r =
  \rho, 2\rho, 3\rho$, . . . such that $P$ is not contained fully
  inside$B(p_i,r)$.

Implement and experiment with this fatness measure. It may be possible to use
the algorithm to assist a project at Harvard on Gerrymandering, where the goal
is to quantify how “compact” polygonal election districts are. (So I hope that
at least a couple of people choose to do this project! We can discuss further.)

(Theoretically, we are interested in finding an algorithm to compute the “exact”
fatness of $P$ (which allows disks centered anywhere inside $P$, of any radius
such that the disk does not contain all of $P$), without resorting to the simple
discretization. Or, can we compute a provable approximation to the exact
fatness?)

\section{Inferences and Questions}
The decomposition of the polygon results in some fat Polygons as well as some bad Polygons. We see that a fat Polygon is made a bad Polygon by inserting the bad polygon. This makes me wonder about the union of the Simple Polygons NICE+BAD, BAD+BAD, NICE+NICE. 
\subsubsection{Questions} 
\begin{enumerate}
\item Q1: Given a Simple Polygon, does it fatness depend on the fatness of the
  individual polygons its made of?
\item Q2: So, for an individual polygon, without loss of generality lets assume that every Simple Polygon is made up of triangles. So, to measure the fatness of a polygon, we  have to measure the fatness of a triangle. So, the next question we ask ourselves is the fatness of any given triangle, the simplest polygon.
\item
Q3: Can we get the fatness of the triangle using the $\alpha$-fatness using the formula given above, whereas is it possible to use squares instead of circles as we know our boundary is a Polygon with straight lines but not a curve.
Q4: About the defnition of the $\alpha-fatness$
Q5: Why do we have to discretize the 
Q6: Alternative measures like, possibly bad unions of nice polygons or visibility kernal ratio. 
Q7: 
\end{enumerate}

\section{Implementation}



\end{document}
