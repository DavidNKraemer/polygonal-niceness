\documentclass{article}

\usepackage[utf8]{inputenc}
\usepackage[]{amsmath}
\usepackage{amsthm}
\usepackage{amssymb}

\newcommand{\RR}{\mathbb{R}}
\newcommand{\abs}[1]{\vert#1\vert}
\DeclareMathOperator*{\argmin}{\textrm{arg\,min}}

\theoremstyle{definition}
\newtheorem{definition}{Definition}[section]

\title{Fatness of Simple Polygons}
\author{Sharmila Duppala \and David Kraemer}
\date{April 2018}

\begin{document}

\maketitle

\section{Problem Statement} 

[From Professor Mitchell's description.]
Computing the Niceness of a Polygonal Shape. There are various notions of
quantifying how “nice” or how “fat” a simple polygon $P$ is. A “nicest” polygon
might be a regular $n$-gon, which most closely approximates a circular disk. This
project seeks to implement some precise metrics for niceness, and compare them
on simple polygons (possibly moused in by a user or read in from a file, etc).
To make it simple and discrete, I propose that you discretize the boundary of
$P$ with discrete points $p_i$ (additional vertices), with a prescribed spacing,
$\delta$. (That is, for edges of $P$ longer than $\delta$, add vertices along
  the edge so that the new sub-segments are of length at most $\delta$, for a
user-specified parameter $\delta$.)

Then, consider discrete choices of radii $r = \rho, 2\rho, 3\rho, \ldots$ for
disks, $B(p_i,r)$, centered at the discrete boundary points $p_i$, for each
radius $r$. The (discrete) fatness of $P$ is given by the smallest value of the
ratio $\lambda (B(p_i,r)\cap P)/\lambda (B(p_i, r)$, over all choices of $p_i$ and $r =
  \rho, 2\rho, 3\rho, \ldots$ such that $P$ is not contained fully
  inside$B(p_i,r)$.

Implement and experiment with this fatness measure. It may be possible to use
the algorithm to assist a project at Harvard on Gerrymandering, where the goal
is to quantify how “compact” polygonal election districts are. (So I hope that
at least a couple of people choose to do this project! We can discuss further.)

(Theoretically, we are interested in finding an algorithm to compute the “exact”
fatness of $P$ (which allows disks centered anywhere inside $P$, of any radius
such that the disk does not contain all of $P$), without resorting to the simple
discretization. Or, can we compute a provable approximation to the exact
fatness?)

\section{Definitions}

Our present objective is to develop rigorous measures of the ``niceness'' of
simple bounded closed planar polygons with the intuitive hierarchy that regular
$n$-gons are ``nicest'' followed by convex $n$-gons and proceeded by ``fat''
polygons---i.e., polygons which retain their structure under smoothing filters.

Throughout we assume that $P \subseteq \RR^2$ is a simple bounded closed
polygon. We shall denote by $\partial P$ the boundary of $P$, with
$\abs{\partial P}$ denoting the perimeter and $\lambda(P)$ denoting the area.
For given points $x,y \in \RR$, denote the closed line segment bounded by $x$
and $y$ by $[x,y]$.  The class of such polygons is denoted by $\mathcal{P}$.

One class of ``fatness'' measures arises through partitioning $P$ into two
(interior disjoint) polygons $P = P' \cup P''$ via chords.  A chord is a pair
$x,y \in \partial P$ such that the segment $[x,y]$ is contained by the relative
interior of $P$ except at the endpoints $x$ and $y$.  Such a chord $[x,y]$
defines a partition of $P = P' \cup P''$ by orientation, so that $\partial P'$
is the arc from $x$ to $y$ with $[y,x]$ and that $\partial P''$ is the arc from
$y$ to $x$ with $[x,y]$. 

\begin{definition}
  Let $f : \mathcal{P} \to \RR$. For a given $P \in \mathcal{P}$,
  its \emph{chord-$f$ score} is given by
  \begin{equation*}
    s_f(P) = \min_{x,y \in P} \max(f(P'), f(P''))
  \end{equation*}
  where the chord $[x,y]$ partitions $P = P' \cup P''$.
\end{definition}

Intuitively, $f$ represents a polygon's cost with respect to a certain
measurement.  For example, if $f$ indicates the perimeter of $P$, then $s_f$ is
associated with a partition of $P$ such that the maximum perimeter of either
subpolygon is minimized. Indeed, measures such as perimeter or area are the
typical choices of $f$, but any suitable property of the polygon may be
employed.

The visibility kernel of $P$, denoted by $\ker P$, is the subset of $P$ which
``sees'' all of $P$. That is, if $x \in \ker P$ and $y \in P$, then $[x,y]
\subseteq P$. Another proposed measure of polygonal ``niceness'' utilizes the
visibility kernel.

\begin{definition}
  The \emph{visibility kernel score} is given by
  \begin{equation*}
    s_{\textrm{vis}}(P) = \frac{\lambda(\ker P)}{\lambda(P)}.
  \end{equation*}
\end{definition}

If $P$ is a regular $n$-gon or convex $s_{\textrm{vis}}(P) = 1$, since trivially
$\ker P = P$. On the other hand, if $P$ is not star shaped, then
$s_{\textrm{vis}}(P) = 0$.

%\section{Inferences and Questions}
%The decomposition of the polygon results in some fat Polygons as well as some bad Polygons. We see that a fat Polygon is made a bad Polygon by inserting the bad polygon. This makes me wonder about the union of the Simple Polygons NICE+BAD, BAD+BAD, NICE+NICE. 
%\subsubsection{Questions} 
%\begin{enumerate}
%\item Q1: Given a Simple Polygon, does it fatness depend on the fatness of the
%  individual polygons its made of?
%\item Q2: So, for an individual polygon, without loss of generality lets assume that every Simple Polygon is made up of triangles. So, to measure the fatness of a polygon, we  have to measure the fatness of a triangle. So, the next question we ask ourselves is the fatness of any given triangle, the simplest polygon.
%\item
%Q3: Can we get the fatness of the triangle using the $\alpha$-fatness using the formula given above, whereas is it possible to use squares instead of circles as we know our boundary is a Polygon with straight lines but not a curve.
%Q4: About the defnition of the $\alpha-fatness$
%Q5: Why do we have to discretize the 
%Q6: Alternative measures like, possibly bad unions of nice polygons or visibility kernal ratio. 
%Q7: 
%\end{enumerate}

\section{Implementation}

We plan to implement these measures of polygonal ``fatness'' and compare them
for assorted collections of polygons, from theoretical and
randomly-generated examples to historical electoral map data. We hope to
determine the relative strengths of these measures as heuristics for polygonal
``fatness'' individually or in an ensemble on this data. 



\end{document}
